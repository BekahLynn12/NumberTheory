\documentclass[11pt]{article} 
\usepackage[margin=1in]{geometry} 
\usepackage{wrapfig, amsmath,amsthm,amssymb, graphicx, multicol, array,seqsplit}
\usepackage{microtype,numprint}
\usepackage{siunitx}
\usepackage{paracol}
\usepackage{blkarray,multirow,multicol,booktabs,nicematrix}
\usepackage{hhline}
\usepackage{xfrac}
\usepackage{xcolor}
\usepackage[makeroom]{cancel}
\usepackage{listings}
\usepackage{mathtools}
\setlength{\tabcolsep}{0pt}

\newcommand{\N}{\mathbb{N}}
\newcommand{\Z}{\mathbb{Z}}
\newcommand{\F}{\mathbb{F}}
\newcommand{\R}{\mathbb{R}}
\newcommand{\C}{\mathbb{C}}
\newcommand{\Q}{\mathbb{Q}}
\newcommand{\bline}{\noindent\rule[0.5ex]{\linewidth}{1pt}}
\newcommand{\ndiv}{\nmid}
\newcommand{\nequiv}{\not\equiv}


\definecolor{codegreen}{rgb}{0,0.6,0}
\definecolor{codegray}{rgb}{0.5,0.5,0.5}
\definecolor{codepurple}{rgb}{0.58,0,0.82}
\definecolor{backcolour}{rgb}{0.95,0.95,0.92}

%Code listing style named "mystyle"
\lstdefinestyle{mystyle}{
  backgroundcolor=\color{backcolour}, commentstyle=\color{codegreen},
  keywordstyle=\color{magenta},
  numberstyle=\tiny\color{codegray},
  stringstyle=\color{codepurple},
  basicstyle=\ttfamily\footnotesize,
  breakatwhitespace=false,         
  breaklines=true,                 
  captionpos=b,                    
  keepspaces=true,                 
  numbers=left,                    
  numbersep=5pt,                  
  showspaces=false,                
  showstringspaces=false,
  showtabs=false,                  
  tabsize=2
}

%"mystyle" code listing set
\lstset{style=mystyle}

 \newcommand\foo[2]{%
    \begin{minipage}{#1}
    \seqsplit{#2}
    \end{minipage}
    }
\newenvironment{myproof}[1][\proofname]{%
  \begin{proof}[#1]$ $\par\nobreak\ignorespaces
}{%
  \end{proof}
}
\newenvironment{problem}[2][Problem]{\begin{trivlist}
\item[\hskip \labelsep {\bfseries #1}\hskip \labelsep {\bfseries #2.}]}{\end{trivlist}}

\newenvironment{myproblem}[1][Problem]{\begin{trivlist}
    \item[\hskip \labelsep {\bfseries #1.}]}{\end{trivlist}}

\newenvironment{solution}
  {\renewcommand\qedsymbol{$~$}\begin{proof}[Solution]$ $\par\nobreak\ignorespaces}
  {\end{proof}}



\begin{document}

\title{Homework 4}
\author{Rebekah Mayne\\
  Math 370, Fall 2024}
\maketitle


\section{(Page 61-62)}

\begin{problem}{3}
Classify the integers $2,3,..., 21$ as abundant, deficient or perfect.
\end{problem}

\begin{solution}
  \setlength{\tabcolsep}{2pt}
  {\small\begin{tabular}{l|l||l|l||l|l}
      \multicolumn{2}{c||}{\textbf{Deficient}} & \multicolumn{2}{c||}{\textbf{Perfect}}      & \multicolumn{2}{c}{\textbf{Abundant}}                                                                                                       \\ \hline
      2                                        & prime                                       &                                       &                                                &    &                                               \\
      3                                        & prime                                       &                                       &                                                &    &                                               \\
      4                                        & $\sigma(4)=\sigma(2^2)=7<8$                 &                                       &                                                &    &                                               \\
      5                                        & prime                                       &                                       &                                                &    &                                               \\
                                               &                                             & 6                                     & $\sigma(6)=\sigma(2)\cdot \sigma(3) = 12 =12 $ &    &                                               \\
      7                                        & prime                                       &                                       &                                                &    &                                               \\
      8                                        & $\sigma(8)=\sigma(2^3)=15<16$               &                                       &                                                &    &                                               \\
      9                                        & $\sigma(9)=\sigma(3^2)=13<18$               &                                       &                                                &    &                                               \\
      10                                       & $\sigma(10)=\sigma(2)\cdot \sigma(5)=18<20$ &                                       &                                                &    &                                               \\
      11                                       & prime                                       &                                       &                                                &    &                                               \\
                                               &                                             &                                       &                                                & 12 & $\sigma(12)=\sigma(3)\cdot \sigma(2^2)=28>24$ \\
      13                                       & prime                                       &                                       &                                                &    &                                               \\
      14                                       & $\sigma(14)=\sigma(2)\cdot \sigma(7)=24<28$ &                                       &                                                &    &                                               \\
      15                                       & $\sigma(15)=\sigma(3)\cdot \sigma(5)=24<30$ &                                       &                                                &    &                                               \\
      16                                       & $\sigma(16)=\sigma(2^4)=31<32$              &                                       &                                                &    &                                               \\
      17                                       & prime                                       &                                       &                                                &    &                                               \\
                                               & \                                           &                                       &                                                & 18 & $\sigma(18)=\sigma(2)\cdot \sigma(3^2)=39>36$ \\
      19                                       & prime                                       &                                       &                                                &    &                                               \\
                                               &                                             &                                       &                                                & 20 & $\sigma(20)=\sigma(2^2)\cdot \sigma(5)=42>40$ \\
      21                                       & $\sigma(21)=\sigma(3)\cdot \sigma(7)=32<42$ &                                       &                                                &    &                                               \\
    \end{tabular}}
\end{solution}


\begin{problem}{5}
If $\sigma(n)=kn$, then $n$ is called a \textit{k-perfect number}. Verify that 672 is a 3-perfect and $2,178,540=2^2\cdot 3^2\cdot 5\cdot 7^2\cdot 13\cdot 19$ is 4-perfect.
\end{problem}

\begin{solution}
  672 is a 3-perfect because we can see that

  \[
    \sigma(672)=\sigma(3) \cdot \sigma(2^5) \cdot \sigma(7)= 4 \cdot 63 \cdot 8 = 3 \cdot (3 \cdot 7 \cdot 5 \cdot 2^5) = 3(672) \;\; \checkmark
  \]

  The second one we can see
  \begin{align*}
    2,178,540         & =2^2\cdot 3^2\cdot 5\cdot 7^2\cdot 13\cdot 19                                                 \\
    \sigma(2,178,540) & =\sigma(2^2)\cdot \sigma(3^2)\cdot \sigma(5)\cdot \sigma(7^2)\cdot \sigma(13)\cdot \sigma(19) \\
    \sigma(2,178,540) & =7 \cdot 13 \cdot 6 \cdot 57 \cdot 14 \cdot 20                                                \\
    \sigma(2,178,540) & =2^2(2^2 \cdot 3^2\cdot 5 \cdot 7^2 \cdot 13 \cdot  19)                                       \\
    \sigma(2,178,540) & =4(2,178,540) \;\; \checkmark
  \end{align*}
\end{solution}



\begin{problem}{6}
Show that no number of the form $2^a3^b$ is 3-perfect.
\end{problem}

\begin{proof}
  Let $n=2^a 3^b$, for contradiction, let this be 3-perfect, so $\sigma(n)=3n$, so we
  \begin{align*}
    \sigma(n)                                     & = 3n              \\
    \sigma(2^a) \cdot \sigma(3^b)                 & = 2^a 3^{b+1}     \\
    (2^{a+1} -1) \left(\frac{3^{b+1}-1}{2}\right) & = 2^a 3^{b+1}     \\
    (2^{a+1} -1) (3^{b+1}-1)                      & = 2^{a+1} 3^{b+1} \\
    2^{a+1}3^{b+1} - 2^{a+1} -3^{b+1} +1          & = 2^{a+1} 3^{b+1} \\
    - 2^{a+1} -3^{b+1} +1                         & = 0               \\
    2^{a+1} 3^{b+1}                               & = 1               \\
  \end{align*}
  But if this was true, then $a$ and $b$ would both need to be -1, which is not possible. So we can se that no number of the form $2^a3^b$ is 3-perfect.
\end{proof}



\begin{problem}{7}
Let us say that $n$ is \textit{superperfect} if and only if $\sigma(\sigma(n))=2n$. Show that if $n=2^k$ and $2^{k+1}-1$ is prime, then $n$ is superperfect.
\end{problem}

\begin{proof}
  Let $n=2^k$, and let $2^{k+1}-1$ be prime and look at $\sigma(\sigma(n))$,
  \begin{align*}
    \sigma(\sigma(n)) & = \sigma (2^{k+1} -1) \\
                      & = 2^{k+1} -1 + 1      \\
                      & = 2^{k+1}             \\
                      & =  2(2^{k})           \\
    \sigma(\sigma(n)) & =  2n                 \\
  \end{align*}
  This is the definition of superperfect, so we can see that $n$ is superperfect.
\end{proof}




\begin{problem}{13}
Show that all even perfect numbers end in 6 or 8.
\end{problem}

\setlength{\columnseprule}{1pt}
\def\columnseprulecolor{\color{black}}
\begin{proof}
  Let $n$ be any even perfect number. We know from class that every even perfect number is of the form $2^{k-1}(2^k-1)$ where $2^{k}-1$ is prime. In homework 2, we showed that for $2^{k}-1$ to be prime, $k$ must be prime. Lets look at two cases, where $k$ is even, and when $k$ is odd.

  When $k$ is even and prime, $k=2$, so then $n=2^{2-1}(2^2-1)=2(4-1)=6$, so it ends in a 6.

  The other case, when $k$ is odd, means that in $2^{k-1}$ $k-1$ is even. Then look mod 10.
  \[
    \begin{array}{c|c}
      a & 2^{a}\pmod{10} \\
      \hline
      1 & 2 \pmod{10}    \\
      2 & 4 \pmod{10}    \\
      3 & 8 \pmod{10}    \\
      4 & 6 \pmod{10}    \\
      5 & 2 \pmod{10}    \\
    \end{array}
  \]
  We can see this means that $2^{k-1}\equiv 4\pmod{10}$ if $k\equiv 3\pmod{4}$ or $2^{k-1}\equiv 6\pmod{10}$, when $k\equiv 1\pmod{4}$. Then, we can check each case alone,
  \begin{multicols}{2}
    $k\equiv 3\pmod{4}$
    \hrule
    \hspace{1pt}

    We know $2^{k-1}\equiv 4\pmod{10}$, and we can see that $2^k-1 \equiv 7\pmod{10}$ so
    \begin{align*}
      (2^{k-1})(2^k-1) & \equiv 4 \cdot 7 & \pmod{10} \\
      (2^{k-1})(2^k-1) & \equiv 8         & \pmod{10}
    \end{align*}

    \columnbreak
    $k\equiv 1\pmod{4}$
    \hrule
    \hspace{1pt}

    We know $2^{k-1}\equiv 6\pmod{10}$, and we can see that $2^k-1 \equiv 1\pmod{10}$ so
    \begin{align*}
      (2^{k-1})(2^k-1) & \equiv 6 \cdot 1 & \pmod{10} \\
      (2^{k-1})(2^k-1) & \equiv 6         & \pmod{10}
    \end{align*}
  \end{multicols}
\end{proof}




\begin{problem}{14}
If $n$ is an even perfect number and $n>6$, show that the sum of its digits is congruent to $1\pmod{9}$.
\end{problem}

\begin{proof}
  Let $n$ be any even perfect number. We know from class that every even perfect number is of the form $2^{k-1}(2^k-1)$ where $2^{k}-1$ is prime. We also know that the sum of the digits of a number is equal to what that number is congruent to mod 9.

  We know that $n=2^{k-1}(2^k-1)$ creates $n=6$ when $k=2$, so because $n>6$ we know $k>2$, and we know that $k$ will be odd.
  Then, lets look at this mod 9.
  We can see that $2^k\pmod{9}$ will repeat mod 6, as follows
  \[
    \begin{array}{c|c}
      a & 2^{a}\pmod{9} \\
      \hline
      1 & 2 \pmod{9}    \\
      2 & 4 \pmod{9}    \\
      3 & 8 \pmod{9}    \\
      4 & 7 \pmod{9}    \\
      5 & 5 \pmod{9}    \\
      6 & 1 \pmod{9}    \\
    \end{array}
  \]
  So, then
  \[
    \begin{array}{c|c|c|c}
      k \pmod{6} & 2^{k-1} \pmod{9} & (2^k-1)\pmod{9} & n\pmod{9} \\
      \hline
      3          & 4                & 7               & 1         \\
      5          & 7                & 4               & 1         \\
      1          & 1                & 1               & 1         \\
    \end{array}
  \]
  So we can see that it is always congruent to 1 mod 9.
\end{proof}

\hrule
~\newline

For problems (2)-(4), the ``daughter" function $F$ is given, and you are asked to find the ``mother," $f$. That is, given the function $F(n)$, find the function $f(n)$ satisfying
\[
  F(n) = \sum_{d|n} f(d)
\]
You may find it easiest to define $f$ piecewise but you may define it however you wish, just make sure that your definition is clear from the PPF of the input.


%2
\section{}
\begin{myproblem}
  Find the ``mother'' of $F(n)=\mu(n)$
\end{myproblem}

\begin{solution}
  Starting with cases, we can see that we have
  \[
    \mu(n) = \sum_{d|n} f(n)
  \]

  \[
    \begin{matrix*}[l]
      n=1                                                                \\
      & \mu(1)   & = 1  & = f(1)                        & f(1)   & =1   \\
      n=p                                                                \\
      & \mu(p)   & = -1 & = f(1)+f(p)                   & f(p)   & = -2 \\
      n=p^2                                                              \\
      & \mu(p^2) & = 0  & = f(1)+f(p) + f(p^2)          & f(p^2) & = 1  \\
      n=p^3                                                              \\
      & \mu(p^3) & = 0  & = f(1)+f(p) + f(p^2) + f(p^3) & f(p^3) & = 0  \\
    \end{matrix*}
  \]
  We can see the pattern of this means that
  \[
    f(p^k)=\begin{cases}
      1  & \text{if $k=2$ or if $p^k=1$} \\
      -2 & \text{if $k=1$}               \\
      0  & \text{if $k>2$}
    \end{cases}
  \]
  Then, overall we can say that
  \[
    f(n)=\begin{cases}
      1      & \text{if $n=1$}                                          \\
      (-2)^k & \text{if $n=p_1\cdots p_k \cdot p_{k+1}^2 \cdots p_r^2$} \\
      0      & \text{if $p^3|n$}
    \end{cases}
  \]
\end{solution}



%3
\section{}
\begin{myproblem}
  Find the ``mother'' of $F(n)=(\mu(n))^2$
\end{myproblem}

\begin{solution}
  \begin{align*}
    \mu * (\mu(n))^2   & = \sum_{d|n} (\mu(d))^2 \cdot \mu\left(\frac{n}{d}\right)                                                             \\
    \intertext{We can just look at $p^k$}
    \mu * (\mu(p^k))^2 & = \sum_{i=0}^{k} (\mu(p^i))^2 \cdot \mu(p^{k-i})                                                                      \\
                       & = \mu(1)^2\mu(p^k) + \mu(p)^2\mu(p^{k-1}) + \mu(p^2)^2\mu(p^{k-2}) + \cdots + \mu(p^{k-1})^2\mu(p) + \mu(p^k)^2\mu(1) \\
                       & = \mu(1)^2\mu(p^k) + \mu(p)^2\mu(p^{k-1}) + \mu(p^2)^2\mu(p^{k-2}) + \cdots + \mu(p^{k-1})^2\mu(p) + \mu(p^k)^2\mu(1) \\
    \mu * (\mu(p^k))^2 & = \begin{cases}
                             -1 & \text{for k=2}   \\
                             0  & \text{otherwise}
                           \end{cases}
    \intertext{Then this extends to}
    \mu * (\mu(n))^2   & = \begin{cases}
                             1      & \text{when $n=1$}                      \\
                             (-1)^k & \text{when $n=p_1^2p_2^2\cdots p_k^2$} \\
                             0      & \text{anything else}                   \\
                           \end{cases}
  \end{align*}

\end{solution}


%4
\section{}
\begin{myproblem}
  Find the ``mother'' of $F(n)=1$ if $n$ is odd, and 0 if $n$ is even.
\end{myproblem}

\begin{solution}
  \begin{align*}
    F * \mu(n)   & = \sum_{d|n} F(d) \cdot \mu\left(\frac{n}{d}\right)               \\
    \intertext{We can just look at $p^k$ first assuming $p$ is odd}
    F * \mu(p^k) & = \sum_{i=0}^k F(p^i) \cdot \mu(p^{k-i})                          \\
    F * \mu(p^k) & = \sum_{i=0}^k 1 \cdot \mu(p^{k-i})                               \\
    F * \mu(p^k) & = \begin{cases}
                       -1 & \text{when $k=1$}        \\
                       0  & \text{for anything else} \\
                     \end{cases}                                   \\
    \intertext{ If $p$ is ever even, then it is 0, so generalized}
    F * \mu(n)   & = \begin{cases}
                       1      & \text{when $n=1$}                                      \\
                       (-1)^k & \text{when $n=p_1p_2\cdots p_k$ where all $p_i\neq 2$} \\
                       0      & \text{if $2|n$ or $p^2|n$ for any $p$}
                     \end{cases}
  \end{align*}

\end{solution}

\hrule
~\newline

For problems (5)-(9) let $*$ denote the Dirichlet convolution operation. That is
\[
  f * g(n) = \sum_{d|n} f(d) \cdot g\left(\frac{n}{d}\right)
\]
Find the following functions (these should all be functions that you can easily write down or define):


%5
\section{}

\begin{myproblem}
  Find $\phi * \sigma$
\end{myproblem}

\begin{solution}
  \begin{align*}
    \phi * \sigma(n)   & = \sum_{d|n} \phi(d) \cdot \sigma\left(\frac{n}{d}\right)                                                   \\
    \intertext{Look at just $p$}
    \phi * \sigma(p)   & = \phi(p) \cdot \sigma(1)  + \phi(1) \cdot \sigma(p)                                                        \\
                       & = p-1 + \frac{p^{2}-1}{p-1}                                                                                 \\
                       & = p-1 + \frac{(p-1)(p+1)}{p-1}                                                                              \\
                       & = p-1 + p+1                                                                                                 \\
                       & = 2p                                                                                                        \\
    \intertext{Look at just $p^2$}
    \phi * \sigma(p^2) & = \sum_{d|p^2} \phi(d) \cdot \sigma\left(\frac{n}{d}\right)                                                 \\
                       & = \phi(1) \cdot \sigma(p^2)+ \phi(p) \cdot \sigma(p)+ \phi(p^2) \cdot \sigma(1)                             \\
                       & =  \frac{p^3-1}{p-1} + (p-1) \cdot \frac{p^{2}-1}{p-1} + p(p-1)                                             \\
                       & =  p^2 + p + 1 + p^{2}-1 + p^2 - p                                                                          \\
                       & =  3p^2                                                                                                     \\
    \intertext{Look at just $p^3$}
    \phi * \sigma(p^3) & = \sum_{d|p^3} \phi(d) \cdot \sigma\left(\frac{n}{d}\right)                                                 \\
                       & = \phi(1) \cdot \sigma(p^3)+ \phi(p) \cdot \sigma(p^2)+\phi(p^2) \cdot \sigma(p) +\phi(p^3) \cdot \sigma(1) \\
                       & =  \frac{p^{4}-1}{p-1} + (p-1)\cdot \frac{p^{3}-1}{p-1} + p(p-1)\cdot \frac{p^{2}-1}{p-1} + p^{2}(p-1)      \\
                       & =  \frac{p^{4}-1}{p-1} + (p^{3}-1) + p\cdot (p^{2}-1) + p^{2}(p-1)                                          \\
                       & =  \frac{p^{4}-1}{p-1} + p^{3}-1 + p^{3}-p + p^3-p^2                                                        \\
                       & =  p^3+p^2+p+1 + 3p^{3}-p^2 -p-1                                                                            \\
                       & =  4p^3                                                                                                     \\
  \end{align*}
  We can see that $\phi * \sigma(p^k)=(k+1)p^k$. So we can write that when $n=p_1^{e_1}\cdot p_2^{e_2}\cdots p_k^{e_k}$
  \[
    \phi * \sigma(n) = \prod_{i=1}^{k} (e_i+1)p_i^{e_i}
  \]
  Or we could also see it as $\tau(n)\cdot N(n)$.
\end{solution}


%6
\section{}

\begin{myproblem}
  Find $\tau * \phi$
\end{myproblem}

\begin{solution}
  \begin{align*}
    \tau * \phi(n)   & = \sum_{d|n} \tau(d) \cdot \phi\left(\frac{n}{d}\right)                                                   \\
    \intertext{Look at just $p$}
    \tau * \phi(p)   & =  \tau(1) \cdot \phi(p) + \tau(p) \cdot \phi(1)                                                          \\
                     & = (p-1) + (1+1)                                                                                           \\
                     & = p + 1                                                                                                   \\
    \intertext{Look at just $p^2$}
    \tau * \phi(p^2) & =  \tau(1) \cdot \phi(p^2) +\tau(p) \cdot \phi(p)  +  \tau(p^2) \cdot \phi(1)                             \\
                     & = p(p-1) + (1+1)(p-1) + (2+1)                                                                             \\
                     & = p^2-p + 2p-2 + 3                                                                                        \\
                     & = p^2+p+1                                                                                                 \\
    \intertext{Look at just $p^3$}
    \tau * \phi(p^2) & =  \tau(1) \cdot \phi(p^3) +\tau(p) \cdot \phi(p^2) + \tau(p^2) \cdot \phi(p)  +  \tau(p^3) \cdot \phi(1) \\
                     & = p^2(p-1) + (2)(p(p-1)) + (3)(p-1) + (4)                                                                 \\
                     & = p^3-p^2 + 2p^2-2p + 3p-3 + 4                                                                            \\
                     & = p^3+p^2+p+1                                                                                             \\
  \end{align*}
  So we can see that $\tau * \phi(p^k)=\sum_{i=0}^k p^{i}$. So we can write that when $n=p_1^{e_1}\cdot p_2^{e_2}\cdots p_k^{e_k}$
  \[
    \tau * \phi(n) = \prod_{i=1}^{k} \sum_{a=0}^{e_i} p_i^{a}
  \]
  However, this is the same as $\sigma(n)$.
\end{solution}


%7
\section{}

\begin{myproblem}
  Find $N * N$
\end{myproblem}

\begin{solution}
  \begin{align*}
    N * N(n) & = \sum_{d|n} N(d) \cdot N\left(\frac{n}{d}\right) \\
    \intertext{Look at just $p^k$}
    N * N(p) & =  \sum_{i=0}^k N(p^{i}) \cdot N(p^{k-i})         \\
             & =  \sum_{i=0}^k p^{i} \cdot p^{k-i}               \\
             & =  \sum_{i=0}^k p^{k}                             \\
             & =  (k+1) p^{k}                                    \\
  \end{align*}
  So we can write that when $n=p_1^{e_1}\cdot p_2^{e_2}\cdots p_k^{e_k}$
  \[
    N * N(n) = \prod_{i=1}^{k} (e_i+1)p_i^{e_i}
  \]
  Or we could also see it as $\tau(n)\cdot N(n)$.
\end{solution}


%8
\section{}

\begin{myproblem}
  Find $\phi * 1$
\end{myproblem}

\begin{solution}
  \begin{align*}
    \phi * 1(n)   & = \sum_{d|n} \phi(d) \cdot 1\left(\frac{n}{d}\right) \\
    \intertext{Look at just $p^k$}
    \phi * 1(p^k) & =  \sum_{d|p^k} \phi(d) \cdot 1                      \\
                  & =  \phi(1) + \phi(p) + \phi(p^2) +\cdots + \phi(p^k) \\
                  & =1 + (p-1) + (p^2-p) + \cdots + (p^k-p^{k-1})        \\
                  & =p^k                                                 \\
  \end{align*}
  So we can see that $\phi * 1(n)=n$ or simply the function $N$.
\end{solution}



%9
\section{}

\begin{myproblem}
  Find $\sigma * \mu$
\end{myproblem}

\begin{solution}
  \begin{align*}
    \sigma * \mu(n)   & = \sum_{d|n} \sigma(d) \cdot \mu\left(\frac{n}{d}\right)                                                    \\
    \intertext{Look at just $p$}
    \sigma * \mu(p)   & =  \sum_{d|p} \sigma(d) \cdot \mu\left(\frac{n}{d}\right)                                                   \\
                      & =  \sigma(1) \cdot \mu(p) + \sigma(p) \cdot \mu(1)                                                          \\
                      & = -1 + \frac{p^2-1}{p-1}                                                                                    \\
                      & = -1 + p + 1                                                                                                \\
                      & = p                                                                                                         \\
    \intertext{Look at just $p^2$}
    \sigma * \mu(p^2) & =  \sum_{d|p^2} \sigma(d) \cdot \mu\left(\frac{n}{d}\right)                                                 \\
                      & =  \sigma(1) \cdot \mu(p^2) + \sigma(p) \cdot \mu(p) + \sigma(p^2) \cdot \mu(1)                             \\
                      & = 0  +  -1 \cdot \frac{p^2-1}{p-1} + \frac{p^3-1}{p-1}                                                      \\
                      & = -(p+1) + (p^2+p+1)                                                                                        \\
                      & = p^2
    \intertext{Look at just $p^3$}
    \sigma * \mu(p^3) & =  \sum_{d|p^3} \sigma(d) \cdot \mu\left(\frac{n}{d}\right)                                                 \\
                      & =  \sigma(1) \cdot \mu(p^3) + \sigma(p) \cdot \mu(p^2) +\sigma(p^2) \cdot \mu(p)+  \sigma(p^3) \cdot \mu(1) \\
                      & = 0 + 0 + \frac{p^3-1}{p-1} \cdot (-1) + \frac{p^4-1}{p-1}                                                  \\
                      & = -(p^2+p+1)+(p^3+p^2+p+1)                                                                                  \\
                      & = p^3
  \end{align*}
  So we can see that $\sigma * \mu(p^k)=p^k$, so $\sigma * \mu(n)=n$ or the function $N$.
\end{solution}

\hrule

%10 
\section{}

\begin{myproblem}
  Prove that there are infinitely many integers $n$ such that $\mu(n) + \mu(n+1)=0$.
\end{myproblem}

\begin{proof}
  Assume to the contrary that we know all of the integers $n$ such that $\mu(n) + \mu(n+1)=0$ and that they are finite, so that $a$ is the largest possible integer where this is true. Then let $c>a$, and let $c+1$ be the square of any odd prime. meaning $c+1 = p^2$. Then, looking at this mod 4, we know that $p\equiv 1\pmod{4}$ or $p\equiv 3\pmod{4}$, and then $p^2=1\pmod{4}$ either way. Then,
  \begin{align*}
    c+1 & = 1 & \pmod{4} \\
    c   & = 0 & \pmod{4} \\
  \end{align*}
  Meaning that $c$ is divisible by 4, and therefore $\mu(c)=0$, but we already knew that $\mu(c+1)=0$, meaning that we know that $a$ is not the largest, and therefore the set of integers is infinite.
\end{proof}



%11
\section{}

\begin{myproblem}
  Prove that there are infinitely many integers $n$ such that $\mu(n)+\mu(n+1)+\mu(n+2)=0$.
\end{myproblem}

\begin{proof}
  We can look at when they are all 0 due to a unique repeated prime factor in each, so we can create
  \begin{align*}
    n \equiv 0 \pmod{p_1^2}   \\
    n+1 \equiv 0 \pmod{p_2^2} \\
    n+2 \equiv 0 \pmod{p_3^2}
  \end{align*}
  By the Chinese Remainder Theorem, there exists a solution mod $p_1^2p_2^2p_3^2$, so we know that there are infinitely many solutions of integers.
\end{proof}


%12
\section{}

\begin{myproblem}
  Prove that there are infinitely many integers $n$ such that $\mu(n)+\mu(n+1)=-1$.
\end{myproblem}

\begin{proof}
  \textbf{Not Finished}
\end{proof}


%13
\section{}

\begin{myproblem}
  Prove that
  \[
    \sum_{d|n} \frac{(\mu(d))^2}{\phi(d)} = \frac{n}{\phi(n)}
  \]
\end{myproblem}

\begin{proof}
  We want to start with
  \[
    \sum_{d|n} \frac{(\mu(d))^2}{\phi(d)}
  \]
  Any divisors with powers over 2 will be canceled by $\mu(d)=0$, so we only need to look at the divisors of the form $p_1\cdots p_k$. Then, since $\mu(d)$ is squared, all of the numerators we need to look at will be 1. So all we need to look at is
  \[
    \sum_{d^{\star}|n} \frac{1}{\phi(d)} \;\; \text{$d^{\star}$ is $d$ of the form $p_1\cdots p_k$}
  \]
  \textbf{Not Finished}
\end{proof}


%14
\section{}

\begin{myproblem}
  Use Sage to find
  \begin{itemize}
    \item[(a)] The first 10 abundant numbers
    \item[(b)] The relative frequency of abundance for the first $n$ positive integers, for $n=$100, 1000, 10000, 100000. That is, how many integers (out of $n$) are abundant.
  \end{itemize}
\end{myproblem}

\begin{solution}
  \begin{itemize}
    \item [(a)]
          \lstinputlisting[language=Python, firstline=2, lastline =24]{Homework4.md}
    \item [(b)]
    \item \lstinputlisting[language=Python, firstline=28, lastline =59]{Homework4.md}
  \end{itemize}
\end{solution}


\hrule

%15
\section{(5 pts Extra Credit)}

\begin{myproblem}
  A positive integer $n$ is ``perfectly crazy'' if $\phi(n)^{\sigma(n)^{\tau(n)}} = n^2$. Find all perfectly crazy numbers.
\end{myproblem}

\begin{solution}

\end{solution}


%16
\section{(5 pts Extra Credit)}

\begin{myproblem}
  Let $P(n)$ be the product of the positive integers which are $\leq n$ and relative prime to $n$. Prove that
  \[
    P(n) = n^{\phi(n)} \prod_{d|n} \left(\frac{d!}{d^d}\right)^{\mu(n/d)}
  \]
\end{myproblem}

\begin{proof}

\end{proof}


\end{document}