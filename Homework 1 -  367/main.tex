\documentclass[11pt]{article} 
\usepackage[margin=1in]{geometry} 
\usepackage{wrapfig, amsmath,amsthm,amssymb, graphicx, multicol, array,seqsplit}
\usepackage{microtype,numprint}
\usepackage{siunitx}
\usepackage{paracol}
\usepackage{blkarray,multirow,multicol,booktabs}
\usepackage{hhline}
\usepackage{xfrac}
\usepackage{xcolor}
\usepackage[makeroom]{cancel}
\usepackage{listings}
\setlength{\tabcolsep}{0pt}

\newcommand{\N}{\mathbb{N}}
\newcommand{\Z}{\mathbb{Z}}
\newcommand{\F}{\mathbb{F}}
\newcommand{\R}{\mathbb{R}}
\newcommand{\C}{\mathbb{C}}

\definecolor{codegreen}{rgb}{0,0.6,0}
\definecolor{codegray}{rgb}{0.5,0.5,0.5}
\definecolor{codepurple}{rgb}{0.58,0,0.82}
\definecolor{backcolour}{rgb}{0.95,0.95,0.92}

%Code listing style named "mystyle"
\lstdefinestyle{mystyle}{
  backgroundcolor=\color{backcolour}, commentstyle=\color{codegreen},
  keywordstyle=\color{magenta},
  numberstyle=\tiny\color{codegray},
  stringstyle=\color{codepurple},
  basicstyle=\ttfamily\footnotesize,
  breakatwhitespace=false,         
  breaklines=true,                 
  captionpos=b,                    
  keepspaces=true,                 
  numbers=left,                    
  numbersep=5pt,                  
  showspaces=false,                
  showstringspaces=false,
  showtabs=false,                  
  tabsize=2
}

%"mystyle" code listing set
\lstset{style=mystyle}

 \newcommand\foo[2]{%
    \begin{minipage}{#1}
    \seqsplit{#2}
    \end{minipage}
    }
\newenvironment{myproof}[1][\proofname]{%
  \begin{proof}[#1]$ $\par\nobreak\ignorespaces
}{%
  \end{proof}
}
\newenvironment{problem}[2][Problem]{\begin{trivlist}
\item[\hskip \labelsep {\bfseries #1}\hskip \labelsep {\bfseries #2.}]}{\end{trivlist}}

\newenvironment{solution}
  {\renewcommand\qedsymbol{$~$}\begin{proof}[Solution]$ $\par\nobreak\ignorespaces}
  {\end{proof}}


\begin{document}
 
\title{Homework 12}
\author{Rebekah Mayne\\
Math 370, Fall 2024}
\maketitle


 \section{Class Questions}
 
\begin{problem}{1}
How many zeroes are at the end of $1000!$ 
\end{problem}

\begin{solution}

\end{solution}



\section{Page 9 Problems}

\begin{problem}{2} 
Calculate (3141) and (10001,100083).
\end{problem}

\begin{solution}

\end{solution}


\begin{problem}{4} 
Find $x$ and $y$ such that $4144x+7696y=592$.
\end{problem}

\begin{solution}

\end{solution}



\begin{problem}{5} 
If $N=abc+1$, prove that $(N,a)=(N,b)=(N,c)=1$.
\end{problem}

\begin{myproof}

\end{myproof}


\begin{problem}{6} 
Find two different solutions of $299x+247y=13$.
\end{problem}

\begin{solution}

\end{solution}



\begin{problem}{7} 
Prove that if $a|b$ and $b|a$ then $a=b$ or $a=-b$. 
\end{problem}

\begin{proof}

\end{proof}



\begin{problem}{9} 
Prove that $((a,b),b)=(a,b)$.
\end{problem}

\begin{solution}

\end{solution}



\begin{problem}{12} 
Prove: If $a|b$ and $c|d$, then $ac|bd$. 
\end{problem}

\begin{proof}

\end{proof}




\begin{problem}{13} 
Prove: If $d|a$ and $d|b$ then $d^2|ab$. 
\end{problem}

\begin{proof}

\end{proof}




\begin{problem}{14} 
Prove: If $c|ab$ and $(c,a)=d$, then $c|db$. 
\end{problem}

\begin{proof}

\end{proof}



\begin{problem}{15} 
\begin{itemize}
    \item [(a)] If $x^2+ax+b=0$ has an integer root, show that it divides $b$. 
    \item [(b)] If $x^2+ax+b=0$ has a rational root, show that it is in fact an integer.
\end{itemize}
\end{problem}

\begin{solution}
\begin{itemize}
    \item [(a)]
    \item [(b)]
\end{itemize}
\end{solution}



\section{Page 19 Problems}



\begin{problem}{2} 
Find the prime-power decompositions of 2345, 45670, and 999999999999. (Note that $101|1000001$). 
\end{problem}

\begin{solution}

\end{solution}



\begin{problem}{8} 
If $d|ab$, does it follow that $d|a$ or $d|b$? 
\end{problem}

\begin{solution}

\end{solution}



\begin{problem}{10} 
Prove that $n(n+1)$ is never a square for $n>0$. 
\end{problem}

\begin{proof}

\end{proof}



\section{Sage Work}


\begin{problem}{A} 
How many primes are there less than $10^6$?
\end{problem}

\begin{solution}


%\lstinputlisting[language=Python, caption=Problem A Code, firstline= , lastline= ]{}
\end{solution}



\begin{problem}{B} 
Find $x,y$ such that $2015x+93y=31$
\end{problem}

\begin{solution}


%\lstinputlisting[language=Python, caption=Problem B Code]{}
\end{solution}


\begin{problem}{C (Extra Credit)}
Let $r(n)$ be the number formed by repeating $n$ 1s. For example $r(5)=11111$. Find $\gcd(r(2025),r(103))$.

\end{problem}

\begin{solution}


%\lstinputlisting[language=Python, caption=Problem C Code]{}
\end{solution}



\end{document}