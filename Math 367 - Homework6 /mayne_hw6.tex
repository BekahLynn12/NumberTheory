\documentclass[11pt]{article} 
\usepackage[margin=1in]{geometry} 
\usepackage{wrapfig, amsmath,amsthm,amssymb, graphicx, multicol, array,seqsplit}
\usepackage{microtype,numprint}
\usepackage{siunitx}
\usepackage{paracol}
\usepackage{blkarray,multirow,multicol,booktabs,nicematrix}
\usepackage{hhline}
\usepackage{xfrac}
\usepackage{xcolor}
\usepackage[makeroom]{cancel}
\usepackage{listings}
\usepackage{mathtools}
\setlength{\tabcolsep}{0pt}

\newcommand{\N}{\mathbb{N}}
\newcommand{\Z}{\mathbb{Z}}
\newcommand{\F}{\mathbb{F}}
\newcommand{\R}{\mathbb{R}}
\newcommand{\C}{\mathbb{C}}
\newcommand{\Q}{\mathbb{Q}}
\newcommand{\bline}{\noindent\rule[0.5ex]{\linewidth}{1pt}}
\newcommand{\ndiv}{\nmid}
\newcommand{\nequiv}{\not\equiv}
\newcommand\leg[2]{\left(\frac{#1}{#2}\right)}

\definecolor{codegreen}{rgb}{0,0.6,0}
\definecolor{codegray}{rgb}{0.5,0.5,0.5}
\definecolor{codepurple}{rgb}{0.58,0,0.82}
\definecolor{backcolour}{rgb}{0.95,0.95,0.92}

%Code listing style named "mystyle"
\lstdefinestyle{mystyle}{
  backgroundcolor=\color{backcolour}, commentstyle=\color{codegreen},
  keywordstyle=\color{magenta},
  numberstyle=\tiny\color{codegray},
  stringstyle=\color{codepurple},
  basicstyle=\ttfamily\footnotesize,
  breakatwhitespace=false,         
  breaklines=true,                 
  captionpos=b,                    
  keepspaces=true,                 
  numbers=left,                    
  numbersep=5pt,                  
  showspaces=false,                
  showstringspaces=false,
  showtabs=false,                  
  tabsize=2
}

%"mystyle" code listing set
\lstset{style=mystyle}

 \newcommand\foo[2]{%
    \begin{minipage}{#1}
    \seqsplit{#2}
    \end{minipage}
    }
\newenvironment{myproof}[1][\proofname]{%
  \begin{proof}[#1]$ $\par\nobreak\ignorespaces
}{%
  \end{proof}
}
\newenvironment{problem}[2][Problem]{\begin{trivlist}
\item[\hskip \labelsep {\bfseries #1}\hskip \labelsep {\bfseries #2.}]}{\end{trivlist}}

\newenvironment{myproblem}[1][Problem]{\begin{trivlist}
    \item[\hskip \labelsep {\bfseries #1.}]}{\end{trivlist}}

\newenvironment{solution}
  {\renewcommand\qedsymbol{$~$}\begin{proof}[Solution]$ $\par\nobreak\ignorespaces}
  {\end{proof}}



\begin{document}

\title{Homework 4}
\author{Rebekah Mayne\\
  Math 370, Fall 2024}
\maketitle


\section{(Page 93)}

\begin{problem}{2}
Which of the following congruences have solutions?
\[
  \begin{matrix*}[l]
    a) & x^2\equiv 8 \pmod{53} & \phantom{sp} & b) & x^2\equiv 15\pmod{31}    \\
    c) & x^2\equiv 54 \pmod{7} &              & d) & x^2\equiv 625\pmod{9973} \\
  \end{matrix*}
\]
\end{problem}

\begin{solution}
  \begin{tabular}[]{llll}
    a) & \textbf{ No Solution} & \phantom{space} b)                                                      & \textbf{ No Solution}    \\
       &
    $\begin{matrix*}[l]
         8^{\frac{53-1}{2}} & \equiv 8^{26}             & \pmod{53} \\
         & \equiv (8^2)^13           & \pmod{53} \\
         & \equiv (11)^13            & \pmod{53} \\
         & \equiv 11\cdot (11^3)^4   & \pmod{53} \\
         & \equiv 11\cdot (6)^4      & \pmod{53} \\
         & \equiv 11 \cdot 6 \cdot 4 & \pmod{53} \\
         & \equiv 44 \cdot 6         & \pmod{53} \\
         & \equiv -9 \cdot 6         & \pmod{53} \\
         & \equiv -54                & \pmod{53} \\
         & \equiv -1                 & \pmod{53} \\
       \end{matrix*}$
       &                       & $\begin{matrix*}[l]
                                      15^{\frac{31-1}{2}} & \equiv 15^{15}             & \pmod{31} \\
                                      & \equiv (3375)^5            & \pmod{31} \\
                                      & \equiv (3100+275)^5        & \pmod{31} \\
                                      & \equiv (0+279-4)^5         & \pmod{31} \\
                                      & \equiv (-4)^5              & \pmod{31} \\
                                      & \equiv (-4)^2 \cdot (-4)^3 & \pmod{31} \\
                                      & \equiv 16 \cdot -2         & \pmod{31} \\
                                      & \equiv -32                 & \pmod{31} \\
                                      & \equiv -1                  & \pmod{31} \\
                                    \end{matrix*}$                              \\
    c) & \textbf{ No Solution} & \phantom{space} d)                                                      & \textbf{ Has a Solution} \\
       &
    $\begin{matrix*}[l]
         54^{\frac{7-1}{2}} & \equiv 5^{3}      & \pmod{7} \\
         & \equiv 25 \cdot 5 & \pmod{7} \\
         & \equiv -3 \cdot 5 & \pmod{7} \\
         & \equiv -15        & \pmod{7} \\
         & \equiv -1         & \pmod{7} \\
       \end{matrix*}$
       &                       & 625 is a square already
  \end{tabular}

\end{solution}


\begin{problem}{4}
Find solutions for the congruences in Problem 2 that have them.
\end{problem}

\begin{solution}
  Looking at $x^2\equiv 625 \pmod{9973}$,  we can see that $625=26^2$ so $x=25$ is the solution. Then the other solution is $-25\pmod{9973}\equiv 9948$, so the solutions are, 25 and 9948.
\end{solution}


\begin{problem}{5}
Calculate $\leg{33}{71}$, $\leg{34}{71}$, $\leg{35}{71}$, and $\leg{36}{71}.$
\end{problem}

\begin{solution}
  \renewcommand{\arraystretch}{1.3}
  \begin{NiceTabular}[width=0.95\textwidth]{X[l] !{\qquad} X[l]}
    $\begin{matrix*}[l]
         \leg{33}{71} &= \leg{3}{71}\cdot \leg{11}{71} \\
         &= (-1) \cdot \leg{71}{3} \cdot (-1) \cdot \leg{71}{11} \\
         &= \leg{2}{3} \cdot \leg{7}{11} \\
         &= -1 \cdot \leg{11}{7} \\
         &= -\leg{4}{7} \\
         \leg{33}{71} &= -1 \\
         \phantom{a}
       \end{matrix*}$
     &
    $\begin{matrix*}[l]
         \leg{34}{71} &= \leg{2}{71} \cdot \leg{17}{71} \\
         &= 1 \cdot (-1)\leg{71}{17} \\
         &= -\leg{3}{17} \\
         &=  \leg{17}{3} \\
         &= \leg{2}{3} \\
         &=  -1 \\
         \leg{34}{71} &= 1
       \end{matrix*}$ \\
    \phantom{space}                                 \\
    $\begin{matrix*}[l]
         \leg{35}{71} &=  \leg{5}{71} \cdot \leg{7}{71} \\
         &= -1 \cdot \leg{71}{5} \cdot \leg{71}{7} \\
         &= -1 \cdot \leg{1}{5} \cdot {1}{7} \\
         \leg{35}{71}  &= -1
       \end{matrix*}$
     &
    $\begin{matrix*}[l]
         \leg{36}{71} &= 1 \;\;\; ^*(36=6^2) \\
         ~\\
         ~\\
         ~\\
       \end{matrix*}$
  \end{NiceTabular}
  \renewcommand{\arraystretch}{1}
\end{solution}


\begin{problem}{6}
Calculate $\leg{33}{73}$, $\leg{34}{73}$, $\leg{35}{73}$, and $\leg{36}{73}.$
\end{problem}

\begin{solution}
  \renewcommand{\arraystretch}{1.3}
  \begin{NiceTabular}[width=0.95\textwidth]{X[l] !{\qquad} X[l]}
    $\begin{matrix*}[l]
         \leg{33}{73} &= \leg{11}{73}\cdot \leg{3}{73} \\
         &= \leg{73}{11} \cdot \leg{73}{3} \\
         &= \leg{7}{11} \cdot \leg{1}{3} \\
         &= -1\cdot \leg{11}{7} \\
         &= -1 \cdot \leg{4}{7} \\
         \leg{33}{73} &= -1 \\
       \end{matrix*}$
     &
    $\begin{matrix*}[l]
         \leg{34}{73} &= \leg{2}{73} \cdot \leg{17}{73} \\
         &= 1 \cdot \leg{73}{17} \\
         &= \leg{5}{17} \\
         &= \leg{17}{5} \\
         &= \leg{2}{5} \\
         \leg{34}{73} &= -1
       \end{matrix*}$ \\
    \phantom{space}                                 \\
    $\begin{matrix*}[l]
         \leg{35}{73} &= \leg{5}{73} \cdot \leg{7}{73} \\
         &= \leg{73}{5} \cdot \leg{73}{7} \\
         &= \leg{3}{5} \cdot \leg{3}{7} \\
         &= \leg{5}{3} \cdot (-1) \cdot \leg{7}{3} \\
         &= \leg{2}{3} \cdot (-1) \cdot \leg{1}{3} \\
         &= (-1) \cdot (-1) \\
         \leg{35}{73} &= 1
       \end{matrix*}$
     &
    $\begin{matrix*}[l]
         \leg{36}{73} &= 1 \;\;\; ^*(36=6^2) \\
         ~\\
         ~\\
         ~\\
         ~\\
         ~\\
         ~\\
       \end{matrix*}$
  \end{NiceTabular}
  \renewcommand{\arraystretch}{1}
\end{solution}

\begin{problem}{10}
Calculate $\leg{1356}{2467}$ and $\leg{6531}{2467}$.
\end{problem}

\begin{solution}
  First $1356=2^2\cdot 3\cdot 113$, and $2467 = 2400 + 60 + 4 + 3 \equiv 3 \pmod{4}$, so we can start with
  \begin{align*}
    \leg{1356}{2467} & = \leg{2^2}{2467} \cdot \leg{3}{2467} \cdot \leg{113}{2467} \\
                     & = (-1) \cdot \leg{2467}{3} \cdot \leg{2467}{113}            \\
                     & = (-1) \cdot \leg{1}{3} \cdot \leg{94}{113}                 \\
                     & = (-1) \cdot \leg{2}{113} \cdot \leg{47}{113}               \\
                     & = (-1) \cdot \leg{113}{47}                                  \\
                     & = (-1) \cdot \leg{19}{47}                                   \\
                     & = (-1) \cdot \leg{47}{19}                                   \\
                     & = (-1) \cdot \leg{9}{19}                                    \\
    \leg{1356}{2467} & = -1
  \end{align*}

  Then for the other, $6531=3\cdot 7 \cdot 311$, and again $2467\equiv 3\pmod{4}$, so we can start with
  \begin{align*}
    \leg{6531}{2467} & = \leg{3}{2467} \cdot \leg{7}{2467} \cdot \leg{311}{2467}                                  \\
                     & = (-1) \cdot \leg{2467}{3} \cdot (-1) \cdot \leg{2467}{7} \cdot (-1) \cdot \leg{2467}{311} \\
                     & = (-1) \cdot \leg{1}{3} \cdot \leg{3}{7} \cdot \leg{290}{311}                              \\
                     & = (-1) \cdot (-1) \leg{7}{3} \cdot \leg{2}{311} \cdot \leg{5}{311} \cdot \leg{29}{311}     \\
                     & = \leg{311}{5} \cdot \leg{311}{29}                                                         \\
                     & = \leg{1}{5} \cdot \leg{21}{29}                                                            \\
                     & = \leg{3}{29} \cdot \leg{7}{29}                                                            \\
                     & = \leg{29}{3} \cdot \leg{29}{7}                                                            \\
                     & = \leg{2}{3} \cdot \leg{1}{7}                                                              \\
    \leg{6531}{2467} & = -1
  \end{align*}
\end{solution}




\begin{problem}{11}
Show that if $p=q+4a$ ($p$ and $q$  are odd primes), then $\leg{p}{q}=\leg{a}{q}$
\end{problem}

\begin{proof}
  Let $p$ and $q$ be odd primes such that $p=q+4a$ (for some $a\in \Z$). Then look at
  \begin{align*}
    \leg{p}{q} & = \leg{q+4a}{q}               \\
               & = \leg{4a}{q}                 \\
               & = \leg{4}{q} \cdot \leg{a}{q} \\
               & = 1 \cdot \leg{a}{q}          \\
    \leg{p}{q} & = \leg{a}{q}                  \\
  \end{align*}
\end{proof}



\begin{problem}{16}
Show that if $a$ is a quadratic residue$\pmod{p}$ and $ab\equiv 1\pmod{p}$ then $b$ is a quadratic residue$\pmod{p}$.
\end{problem}

\begin{proof}
  Let $a$ be a quadratic residue$\mod{p}$ and $ab\equiv 1 \pmod{p}$, then we know that $\leg{a}{p}=1$, then we have
  \begin{align*}
    \leg{a}{p} & = 1                           \\
               & = \leg{1}{p}                  \\
    \leg{a}{p} & = \leg{ab}{p}                 \\
    1          & = \leg{ab}{p}                 \\
    1          & = \leg{b}{p} \cdot \leg{a}{p} \\
    1          & = \leg{b}{p}
  \end{align*}
  Which by definition means that $b$ is also a quadratic residue$\pmod{p}$.
\end{proof}



\begin{problem}{17}
Does $x^2\equiv 211\pmod{159}$ have a solution? Note that 159 is not prime.
\end{problem}

\begin{solution}
  Yes, since $221\equiv 1 \pmod{3}$, and $221\equiv -1 \pmod{53}$ and $53\equiv 1\pmod{4}$, so there should be solutions for $x^2\equiv 211 \pmod{3}$ and $x^2\equiv 211 \pmod{53}$. Since $3\bot 53$, there should be a solution for $x^2 \equiv 211 \pmod{159}$.
\end{solution}




\begin{problem}{20}
Suppose that $p=q+4a$ where $p$ and $q$ are odd primes. Show that $\leg{a}{p}=\leg{a}{q}$.
\end{problem}

\begin{proof}
  Let $p$ and $q$ be odd primes, where $p=q+4a$ for some $a\in \Z$. This means that $p\equiv q \pmod{4}$, then there are two cases, one where it is equivalent to 1, and one where it is equivalent to 3.

  If $p\equiv q \equiv 1\pmod{4}$, then
  \begin{align*}
    \leg{p}{q}    & = \leg{q}{p}                  \\
    \leg{q+4a}{q} & = \leg{p-4a}{p}               \\
    \leg{4a}{q}   & = \leg{-4a}{p}                \\
    \leg{a}{q}    & = \leg{-1}{p}\cdot \leg{a}{p} \\
    \leg{a}{q}    & = \leg{a}{p}                  \\
  \end{align*}

  If $p\equiv q \equiv 3\pmod{4}$, then
  \begin{align*}
    \leg{p}{q}    & = -\leg{q}{p}                    \\
    \leg{q+4a}{q} & = -\leg{p-4a}{p}                 \\
    \leg{4a}{q}   & = -\leg{-4a}{p}                  \\
    \leg{a}{q}    & = -\leg{-1}{p}\cdot \leg{a}{p}   \\
    \leg{a}{q}    & = (-1)\cdot (-1)\cdot \leg{a}{p} \\
    \leg{a}{q}    & = \leg{a}{p}                     \\
  \end{align*}

  We can see either way, this is true.
\end{proof}

\hrule
~\newline

\section{pg 104}

\begin{problem}{2}
Show that 3 is a quadratic nonresidue of all Mersenne primes greater than 3.
\end{problem}

\begin{solution}
  Let $a=2^p -1$, and assume $a$ is prime. We want to look at $\leg{3}{a}$, we also know that $p>2$ (if $p=2$, then $a=3$, but we are only worried about Mersenne primes greater than 3), which means that $p$ must be odd. Let's think about what $a$ is mod 4. Since $p>2$, then $a\equiv 3 \pmod{4}$, since $2^p\equiv 0 \pmod{4}$ when $p\geq 2$. So we can use quadratic reciprocity to do the following,
  \begin{align*}
    \leg{3}{a} & = -\leg{a}{3}                                          \\
               & \equiv -\left( 2^p - 1\right)^{\frac{3-1}{2}} \pmod{3} \\
               & \equiv -\left(2^p - 1\right) \pmod{3}                  \\
    \intertext{Because $p$ is odd, we know $2^p\equiv 2 \pmod{3}$}
               & \equiv -(2 - 1) \pmod{3}                               \\
               & \equiv -1 \pmod{3}                                     \\
    \leg{3}{a} & = -1                                                   \\
  \end{align*}
  So we can see that 3 is a quadratic nonresidue for all Mersenne primes greater than 3.
\end{solution}


\begin{problem}{4}
~\\
\begin{itemize}
  \item [(a)] Prove that if $p\equiv 7\pmod{8}$, then $p|\left(2^{\left(\frac{(p-1)}{2}\right)}-1\right)$
  \item [(b)] Find a factor of $2^{83}-1$
\end{itemize}
\end{problem}

\begin{solution}
  \renewcommand\qedsymbol{$\square$}
  \begin{itemize}
    \item [(a)]
          \begin{proof}
            Let $p\equiv 7 \pmod{8}$. Then, based on theorem 2 in chapter 12, we know that $\leg{2}{p}=1$, which is the same as saying $2^{\left(\frac{(p-1)}{2}\right)}-1\equiv 0 \pmod{p}$, which is then also the same as saying $p|\left(2^{\left(\frac{(p-1)}{2}\right)}-1\right)$.
          \end{proof}
    \item [(b)] If we look for a $p$ where $\frac{p-1}{2}=83$, we find $p=167$, and since $167\equiv 7 \pmod{8}$, we can use the above to see that $167|2^{83}$.
  \end{itemize}
  \renewcommand\qedsymbol{$~$}
\end{solution}



\begin{problem}{5}
~\\
\begin{itemize}
  \item [(a)] If $p$ and $q=10p+3$ are odd primes, show that $\leg{p}{q}=\leg{3}{p}$.
  \item [(b)] If $p$ and $q=10p+1$ are odd primes, show that $\leg{p}{q}=\leg{-1}{p}$
\end{itemize}
\end{problem}

\begin{solution}
  \renewcommand\qedsymbol{$\square$}
  \begin{itemize}
    \item [(a)]
          \begin{proof}
            Let $p$ and $q=10p+3$ be odd primes. If $p\equiv 3\pmod{4}$, then
            \begin{align*}
              q & \equiv 10(3)+3 \pmod{4} \\
              q & \equiv 33 \pmod{4}      \\
              q & \equiv 1 \pmod{4}       \\
            \end{align*}
            So no matter what at least one of $p$ or $q$ will be not equivalent to 3 mod 4, so we know that we can apply quadratic reciprocity as follows,
            \begin{align*}
              \leg{p}{q} & = \leg{q}{p}     \\
                         & = \leg{10p+3}{p} \\
              \leg{p}{q} & = \leg{3}{p}
            \end{align*}
          \end{proof}
    \item [(b)] \begin{proof}
            Let $p$ and $q=10p+1$ be odd primes. If $p\equiv 3\pmod{4}$, then
            \begin{align*}
              q & \equiv 10(3)+1 \pmod{4} \\
              q & \equiv 31 \pmod{4}      \\
              q & \equiv 3 \pmod{4}       \\
            \end{align*}
            So when $p\equiv 3\pmod{4}$, we have
            \begin{align*}
              \leg{p}{q} & = -\leg{q}{p}     \\
                         & = -\leg{10p+1}{p} \\
              \leg{p}{q} & = -\leg{1}{p}     \\
              \leg{p}{q} & = -1              \\
            \end{align*}
            If $p\equiv 1\pmod{4}$, we have
            \begin{align*}
              \leg{p}{q} & = \leg{q}{p}     \\
                         & = \leg{10p+1}{p} \\
              \leg{p}{q} & = \leg{1}{p}     \\
              \leg{p}{q} & = 1              \\
            \end{align*}
            Which we can see is the same definition as $\leg{-1}{p}$, so $\leg{p}{q}=\leg{-1}{p}$
          \end{proof}
  \end{itemize}
  \renewcommand\qedsymbol{$~$}
\end{solution}


\begin{problem}{6}
~\\
\begin{itemize}
  \item [(a)] Which primes can divide $n^2+1$ for some $n$?
  \item [(b)] Which odd primes can divide $n^2+n$ for some $n$?
  \item [(c)] Which odd primes can divide $n^2+2n+2$ for some $n$?
\end{itemize}
\end{problem}

\begin{solution}
  \begin{itemize}
    \item [(a)] Let $p$ be some odd prime, we want to know for what $p$ we have $p|n^2+1$ for some $n$. This can be rewritten as finding when
          \begin{align*}
            n^2+1 & \equiv 0 \pmod{p}  \\
            n^2   & \equiv -1 \pmod{p} \\
          \end{align*}
          This can be rewritten as when $\leg{-1}{p}=1$. So an odd prime can divide $n^2+1$ for some $n$ if and only if $p\equiv 1 \pmod{4}$, and when $p=2$, (since $1\equiv -1 \pmod{2}$, so $1^2\equiv -1\pmod{2}$).
    \item [(b)] We are looking for when $p|n^2+n$ for some $n$, this can be rewritten as finding when
          \begin{align*}
            n^2+n & \equiv 0 \pmod{p}  \\
            n^2   & \equiv -n \pmod{p} \\
          \end{align*}
          This can be rewritten as when $\leg{-n}{p}=1$. So an odd prime can divide $n^2+n$, for some $n$ if and only if $\leg{-n}{p}=1$.
    \item [(c)] We are looking for when $p|(n^2+n+2)$ for some $n$, this can be rewritten as finding when
          \begin{align*}
            n^2+n +2 & \equiv 0 \pmod{p}      \\
            n^2      & \equiv -(n+2) \pmod{p} \\
          \end{align*}
          This can be rewritten as when $\leg{-(n+2)}{p}=1$. So an odd prime can divide $n^2+n+2$, for some $n$ if and only if $\leg{-(n+2)}{p}=1$.
  \end{itemize}
\end{solution}

\begin{problem}{7} ~\\
\begin{itemize}
  \item [(a)] Show that if $p\equiv 3 \pmod{4}$ and $a$ is a quadratic residue$\pmod{p}$, then $p-a$ is a quadratic nonresidue$\pmod{p}$.
  \item [(b)] What if $p\equiv 1 \pmod{4}$?
\end{itemize}
\end{problem}

\begin{solution}
  \begin{itemize}
    \item [(a)] Let $p\equiv 3 \pmod{4}$, and let $a$ be a quadratic residue$\pmod{p}$, then look at
          \begin{align*}
            \leg{p-a}{p} & = \leg{-a}{p}                  \\
                         & = \leg{-1}{p} \cdot \leg{a}{p} \\
                         & = \leg{-1}{p}                  \\
            \leg{p-a}{p} & = -1
          \end{align*}
          Which is the definition of $p-a$ being a quadratic nonresidue$\pmod{p}$.
    \item [(b)] We can see that the argument doesn't change up until the last step when evaluating $\leg{-1}{p}$, so if $p\equiv 1 \pmod{4}$, $p-a$ is a quadratic residue$\pmod{p}$.
  \end{itemize}
\end{solution}

\hrule
~\newline

\section{}

\begin{myproblem}
  If $p$ is an odd prime, prove that
  \[
    \sum_{a=1}^{p-1} \leg{a}{p}= 0
  \]
\end{myproblem}

\begin{proof}
  Let $p$ be an odd prime, then we know that for any $a$ (where $a$ is a proper residue mod $p$) has the same square outcome as $-a$, so at most we can have $\frac{p-1}{2}$ quadratic residues for $p$, however, we know that we can't have $a$, $b$, in mod $p$ where $a\nequiv \pm b\pmod{p}$ but $a^2\equiv b^2\pmod{p}$, so we will actually have exactly $\frac{p-1}{2}$ quadratic residues for $p$, meaning we will also have exactly $\frac{p-1}{2}$ quadratic nonresidues. This means that there is an equal number of them that will be 1 and $-1$, so it will sum to 0.
\end{proof}



\hrule
~\newline

\section{}

\begin{myproblem}
  For which primes $p=3,5,7,11,13,17$ does $x^2\equiv -2 \pmod{p}$ have a solution?
  Which primes in general guarantee solutions to this equation? Can you prove it?
\end{myproblem}

\begin{proof}
  $x^2\equiv -2 \pmod{p}$ will have solutions when either both $\leg{2}{p}$ and $\leg{-1}{p}$ are equal to 1 or equal to $-1$. We know that
  \[
    \leg{-1}{p} =
    \begin{cases}
      1  & \text{if } p\equiv 1 \pmod{4} \\
      -1 & \text{if } p\equiv 3 \pmod{4} \\
    \end{cases}
  \]
  And that
  \[
    \leg{2}{p} =
    \begin{cases}
      1  & \text{if } p\equiv 1 \pmod{8} \text{ or }  p\equiv 7 \pmod{8} \\
      -1 & \text{if } p\equiv 3 \pmod{8} \text{ or }  p\equiv 5 \pmod{8} \\
    \end{cases}
  \]
  So we need $p\equiv 1\pmod{4}$ and $p\equiv 1 \pmod{8}$, which can be written as just $p\equiv 1 \pmod{8}$ since if $p=8k+1$, this implies that $p\equiv 1\pmod{4}$ as well, the same argument is true for when $p\equiv 3\pmod{8}$. The other two cases do not work since they flip the polarity when in the other modulus, so -2 is a quadratic residue when $p\equiv 1\pmod{8}$ or when $p\equiv 3 \pmod{8}$. So $x^2\equiv -2 \pmod{p}$ has solutions for $p=3,11,$ and $17$.
\end{proof}



\hrule
~\newline

\section{}

\begin{myproblem}
  Let $p\equiv 1\pmod{4}$ and denote both solutions of $x^2\equiv -1 \pmod{p}$ by $i$ and $-i$.
  Prove or disprove:
  \[
    \text{If } a+bi\equiv 0 \pmod{p} \text{ then } a\equiv b\equiv 0 \pmod{p}
  \]
\end{myproblem}

\begin{proof}
  Let $p\equiv 1\pmod{4}$ and let $i$ and $-i$ be the two solutions of $x^2\equiv -1\pmod{p}$. Then let's look at
  \begin{align*}
    a+bi          & \equiv 0 \pmod{p} \\
    (a+bi)(a-bi)  & \equiv 0 \pmod{p} \\
    a^2 - b^2(-1) & \equiv 0 \pmod{p} \\
    a^2 +b^2      & \equiv 0 \pmod{p} \\
  \end{align*}
  We know that this can be solved for $a\nequiv b\nequiv 0$ because $p\equiv 1 \pmod{4}$, so there exists $c^2\equiv -1$ and we can let $a$ or $b$ be $-1$ and the other the solution. So the statement is false.
\end{proof}


\hrule
~\newline

\section{}

\begin{myproblem}
  If $p\equiv 7\pmod{8}$ and $h=\frac{(p-1)}{2}$ is prime, evaluate $\leg{h}{p}$.
\end{myproblem}

\begin{proof}
  If $p\equiv 7\pmod{8}$, then we know that $\leg{2}{p}=1$, so since $h\bot 2$ (since $h$ is prime) we can see this means that $\leg{h}{p}=\leg{2h}{p}$, so we can do the following
  \begin{align*}
    \leg{h}{p} & = \leg{2h}{p}  \\
               & = \leg{p+1}{p} \\
               & = \leg{1}{p}   \\
    \leg{h}{p} & = 1            \\
  \end{align*}
\end{proof}


\hrule
~\newline

\section{}

\begin{myproblem}
  For the following values of $n$ and $e$, find the magic decoding exponent $d$ \textbf{by hand}. You may use a calculator for large computations if needed but please show your steps.
  \begin{itemize}
    \item[(a)] $n=17$, $e=5$
    \item[(b)] $n=21$, $e=11$
  \end{itemize}
\end{myproblem}

\begin{solution}
  \begin{itemize}
    \item [(a)] We have $y \equiv x^5 \pmod{17}$, and we want to find $d$ s.t. $y^{d} \equiv x \pmod{17}$. Our $n$ is prime, so we want $d$ s.t. $(n-1)t + 1 = ed$ for some $t$ or $5d \equiv 1 \pmod{16}$.
          So let's create a chart for mod 16 as follows:
          \[\begin{array}{c| *{16}{c}}
              a            & 0 & 1 & 2  & 3  & 4 & 5 & 6  & 7 & 8 & 9  & 10 & 11 & 12 & 13 & 14 & 15 \\
              \hline
              5a \pmod{16} & 0 & 5 & 10 & 15 & 4 & 9 & 14 & 3 & 8 & 13 & 2  & 7  & 12 & 1  & 6  & 11 \\
            \end{array}\]
          So we can see that $d=13$. We can check this by seeing that
          \begin{align*}
            (x^5)^{13} & \equiv x^{65}               & \pmod{17} \\
                       & \equiv x^{65}               & \pmod{17} \\
                       & \equiv (x^{16})^{4} \cdot x & \pmod{17} \\
                       & \equiv 1^4 \cdot x          & \pmod{17} \\
            (x^5)^{13} & \equiv x  \;\;\; \checkmark & \pmod{17} \\
          \end{align*}






    \item [(b)] We have $y \equiv x^{11} \pmod{21}$, and we want to find $d$ s.t. $y^{d} \equiv x \pmod{21}$. Our $n$ here is $3\cdot 7$, so here we want to find $\phi(n)t+1 = ed$ for some $t$, or $11d \equiv 1 \pmod{\phi(21)}$. First we need to find $\phi(21)$, which we can find by doing
          \begin{align*}
            \phi(21) & = 21 \left(1-\frac{1}{3}\right)\left(1-\frac{1}{7}\right) \\
                     & = 21 \left(\frac{2}{3}\right)\left(\frac{6}{7}\right)     \\
            \phi(21) & = 12                                                      \\
          \end{align*}

          So let's create a chart for mod 12 as follows using that $11\equiv -1 \pmod{12}$:
          \[\begin{array}{c| *{12}{c}}
              a            & 0 & 1  & 2  & 3 & 4 & 5 & 6 & 7 & 8 & 9 & 10 & 11 \\
              \hline
              -a \pmod{12} & 0 & 11 & 10 & 9 & 8 & 7 & 6 & 5 & 4 & 3 & 2  & 1  \\
            \end{array}\]
          So we can see that $d=11$. We can check this by seeing that
          \begin{align*}
            (x^{11})^{11} & \equiv x^{121}               & \pmod{21} \\
                          & \equiv (x^{12})^{10} \cdot x & \pmod{21} \\
                          & \equiv (1)^{10} \cdot x      & \pmod{21} \\
            (x^{11})^{11} & \equiv x  \;\;\; \checkmark  & \pmod{21} \\
          \end{align*}
  \end{itemize}
\end{solution}



\hrule
~\newline

\section{}

\begin{myproblem}
  Find the rational number, in lowest, terms, given by each of the following continued fractions
  \begin{itemize}
    \item [(a)] [3,2,1]
    \item [(b)] [3,7,15,1]
  \end{itemize}
\end{myproblem}

\begin{solution}
  \begin{itemize}
    \item [(a)]
          \begin{align*}
            [3,2,1] & = 3 + \frac{1}{2+ \frac{1}{1}} \\
                    & = 3 + \frac{1}{3}              \\
                    & =\frac{10}{3}                  \\
          \end{align*}
    \item [(b)]
          \begin{align*}
            [3,7,15,1] & = 3 + \frac{1}{7+\frac{1}{15+\frac{1}{1}}} \\
                       & = 3 + \frac{1}{7+\frac{1}{16}}             \\
                       & = 3 + \frac{1}{\frac{113}{16}}             \\
                       & = 3 + \frac{16}{113}                       \\
            [3,7,15,1] & = \frac{355}{113}                          \\
          \end{align*}
  \end{itemize}
\end{solution}



\hrule
~\newline

\section{}

\begin{myproblem}
  Find the simple continued fraction expansion for the following values:
  \begin{itemize}
    \item [(a)] $\frac{32}{17}$
    \item [(b)] $\sqrt{3}$
  \end{itemize}
\end{myproblem}

\begin{solution}
  \begin{itemize}
    \item [(a)]
          \begin{align*}
            \frac{32}{17} & = 1+ \frac{15}{17}                          \\
                          & = 1 + \frac{1}{\frac{17}{15}}               \\
                          & = 1 + \frac{1}{1 + \frac{2}{15}}            \\
                          & = 1 + \frac{1}{1 + \frac{1}{\frac{15}{2}}}  \\
                          & = 1 + \frac{1}{1 + \frac{1}{7+\frac{1}{2}}} \\
            \frac{32}{17} & = [1,1,7,2]
          \end{align*}
    \item [(b)]
          We can start with $\alpha=\sqrt{3}=\alpha_1$, then $a_1=\lfloor \sqrt{3}\rfloor = 1$

          Then, $\alpha_2=\frac{1}{\sqrt{3}-1} = \frac{\sqrt{3}+1}{2}$, and $a_2=\lfloor\frac{\sqrt{3}+1}{2}\rfloor = 1$.


          Then, $\alpha_3 = \frac{1}{\frac{\sqrt{3}+1}{2} -1 }=\frac{1}{\frac{\sqrt{3}+1-2}{2}}=\frac{2}{\sqrt{3}-1} = \frac{2(\sqrt{3}+1)}{2}= \sqrt{3}+1$ and $a_3=\lfloor \sqrt{3}+1\rfloor = 2$.

          Then, $\alpha_4 = \frac{1}{\sqrt{3}+1-2} =\frac{1}{\sqrt{3}-1}=\alpha_2$, so $a_4=a_2$.

          So $\sqrt{3} =[1,\overline{1,2}]$
  \end{itemize}
\end{solution}


\hrule
~\newline

\section{}

\begin{myproblem}
  Find the exact value of he following continued fractions
  \begin{itemize}
    \item [(a)] $ [1,\overline{2} ] = [ 1,2,2,2,2,\cdots ] $
    \item [(b)] $ [3,\overline{2,6}] = [3,2,6,2,6,2,6,\cdots] $
  \end{itemize}
\end{myproblem}



\begin{solution}
  \begin{itemize}
    \item [(a)] We can use the theorem of $[a,\overline{b}]=\frac{2a-b}{2}+\frac{\sqrt{b^2+4}}{2}$ to see that
          \begin{align*}
            [1,\overline{2} ] & = \frac{2(1)-2}{2}+\frac{\sqrt{2^2+4}}{2} \\
                              & = \frac{\sqrt{8}}{2}                      \\
                              & = \frac{2\sqrt{2}}{2}                     \\
            [1,\overline{2} ] & = \sqrt{2}
          \end{align*}
    \item [(b)] We can start by finding $[\overline{2,6}]$ by letting $a= [\overline{2,6}]$ and letting
          \begin{align*}
            a      & = 2 + \frac{1}{6 + \frac{1}{2 + \frac{1}{6 + \frac{1}{\ddots}}}} \\
            a      & = 2 + \frac{1}{6 + \frac{1}{a}}                                  \\
            a      & = 2 + \frac{1}{\frac{6a + 1}{a}}                                 \\
            a      & = 2 + \frac{a}{6a+1}                                             \\
            a      & = \frac{2(6a+1) +a}{6a+1}                                        \\
            a      & = \frac{13a+2}{6a+1}                                             \\
            6a^2+a & = 13a+2                                                          \\
            0      & = 6a^2 -12a-  2                                                  \\
          \end{align*}
          Then we can use the quadratic formula to get
          \begin{align*}
            a & = \frac{12\pm \sqrt{(-12)^2-4(6)(-2)}}{2(6)} \\
            a & = \frac{12\pm \sqrt{144+48}}{12}             \\
            a & = \frac{12\pm \sqrt{192}}{12}                \\
            a & =  \frac{12\pm 8 \sqrt{3}}{12}               \\
            a & =  \frac{3\pm 2 \sqrt{3}}{3}                 \\
          \end{align*}
          We only care about the positive case here.
          Then, we can see that
          \begin{align*}
            [3,\overline{2,6}] & = 3 + \frac{1}{a}                                                          \\
                               & = 3 + \frac{1}{\frac{3+ 2 \sqrt{3}}{3}  }                                  \\
                               & = 3 + \frac{3}{3+ 2 \sqrt{3}}                                              \\
                               & = \frac{3(3+2\sqrt{3})+3}{3+ 2 \sqrt{3}}                                   \\
                               & = \frac{12+6\sqrt{3}}{3+ 2 \sqrt{3}}                                       \\
                               & = \frac{12+6\sqrt{3}}{3+ 2 \sqrt{3}} \cdot \frac{3-2\sqrt{3}}{3-2\sqrt{3}} \\
                               & = \frac{(12+6\sqrt{3})(3-2\sqrt{3})}{9- 12}                                \\
                               & = \frac{36 - 24\sqrt{3} + 18\sqrt{3} -36}{-3}                              \\
                               & = \frac{-6\sqrt{3}}{-3}                                                    \\
            [3,\overline{2,6}] & = 2 \sqrt{3}                                                               \\
          \end{align*}
  \end{itemize}
\end{solution}



\end{document}