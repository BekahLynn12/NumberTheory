\documentclass[11pt]{article} 
\usepackage[margin=1in]{geometry} 
\usepackage{wrapfig, amsmath,amsthm,amssymb, graphicx, multicol, array,seqsplit}
\usepackage{microtype,numprint}
\usepackage{siunitx}
\usepackage{paracol}
\usepackage{blkarray,multirow,multicol,booktabs}
\usepackage{hhline}
\usepackage{xfrac}
\usepackage{xcolor}
\usepackage[makeroom]{cancel}
\usepackage{listings}
\usepackage{mathtools}
\setlength{\tabcolsep}{0pt}

\newcommand{\N}{\mathbb{N}}
\newcommand{\Z}{\mathbb{Z}}
\newcommand{\F}{\mathbb{F}}
\newcommand{\R}{\mathbb{R}}
\newcommand{\C}{\mathbb{C}}
\newcommand{\Q}{\mathbb{Q}}
\newcommand{\bline}{\noindent\rule[0.5ex]{\linewidth}{1pt}}
\newcommand{\ndiv}{\nmid}

\definecolor{codegreen}{rgb}{0,0.6,0}
\definecolor{codegray}{rgb}{0.5,0.5,0.5}
\definecolor{codepurple}{rgb}{0.58,0,0.82}
\definecolor{backcolour}{rgb}{0.95,0.95,0.92}

%Code listing style named "mystyle"
\lstdefinestyle{mystyle}{
  backgroundcolor=\color{backcolour}, commentstyle=\color{codegreen},
  keywordstyle=\color{magenta},
  numberstyle=\tiny\color{codegray},
  stringstyle=\color{codepurple},
  basicstyle=\ttfamily\footnotesize,
  breakatwhitespace=false,         
  breaklines=true,                 
  captionpos=b,                    
  keepspaces=true,                 
  numbers=left,                    
  numbersep=5pt,                  
  showspaces=false,                
  showstringspaces=false,
  showtabs=false,                  
  tabsize=2
}

%"mystyle" code listing set
\lstset{style=mystyle}

 \newcommand\foo[2]{%
    \begin{minipage}{#1}
    \seqsplit{#2}
    \end{minipage}
    }
\newenvironment{myproof}[1][\proofname]{%
  \begin{proof}[#1]$ $\par\nobreak\ignorespaces
}{%
  \end{proof}
}
\newenvironment{problem}[2][Problem]{\begin{trivlist}
\item[\hskip \labelsep {\bfseries #1}\hskip \labelsep {\bfseries #2.}]}{\end{trivlist}}

\newenvironment{solution}
  {\renewcommand\qedsymbol{$~$}\begin{proof}[Solution]$ $\par\nobreak\ignorespaces}
  {\end{proof}}


\begin{document}

\title{Homework 1}
\author{Rebekah Mayne\\
    Math 370, Fall 2024}
\maketitle


\section{(Page 19)}

\begin{problem}{12}
Let $p$ be the least prime factor of $n$, where $n$ is composite. Prove that if $p>n^{\sfrac{1}{3}}$, then $n/p$ is prime.
\end{problem}

\begin{myproof}
    Let $p$ be the least prime factor of $n$, where $n$ is composite, meaning in this case $p<n$. Let $p>n^{\sfrac{1}{3}}$. We know that $p\cdot k=n$ for some $k$, assume to the contrary that $k$ is composite, so $\exists \; m_1,m_2 \in\Z$ s.t. $m_1\cdot m_2=k$. Then, we have that $p\cdot m_1 \cdot m_2=n$. We also know that $m_1,m_2<p$ since $p$ is the least prime factor. So we have that
    \begin{align*}
        p\cdot m_1 \cdot m_2 & > p^3                                   \\
        p\cdot m_1 \cdot m_2 & > p^3 > \left[n^{\sfrac{1}{3}}\right]^3 \\
        p\cdot m_1 \cdot m_2 & > n                                     \\
        n                    & > n                                     \\
    \end{align*}
    But this is an obvious contradiction, so we know that $k$ must be prime, and by our definition $k=\sfrac{n}{p}$ so we know that $\sfrac{n}{p}$ is prime.
\end{myproof}


\begin{problem}{14}
Prove that if $n$ is composite. then $2^n-1$ is composite.
\end{problem}

\begin{myproof}
    Let $n$ be composite, so $\exists \; p,q$ s.t. $p\cdot q =n$ Let $m=2^p-1$, and lets look at this $\pmod{m}$,
    \begin{align*}
        2^p-1 \pmod{m}                   & \equiv 0 \\
        2^p \pmod{m}                     & \equiv 1 \\
        \left(2^{p}\right)^q \pmod{m}    & \equiv 1 \\
        \left(2^{p}\right)^q -1 \pmod{m} & \equiv 0 \\
    \end{align*}
    This can be rewritten as $m| \left(2^{pq} -1\right)$, or $m| \left(2^{n} -1\right)$ which means that $2^n-1$ is composite as well.
\end{myproof}


\begin{problem}{15}
Is it true that if $2^n-1$ is composite, then $n$ is composite?
\end{problem}

\begin{solution}
    Let $p$ be a divisor of $2^n-1$, this means that
    \begin{align*}
        2^n-1 & \equiv 0 \pmod{p} \\
        2^n   & \equiv 1 \pmod{p} \\
    \end{align*}
    We can see that this means that 2 is it's own inverse in $\mod{p}$, so if $n$ was odd $2^n\equiv 2\pmod{p}$, but since $2^n\equiv 1 \pmod{p}$ we know that $n$ must be even. This means that $n$ is either composite, or $n=2$, so it is nt always true, but if $n\neq 2$ then it is.
\end{solution}

\bline

\section{}
\begin{problem}{}
Find the smallest positive integer $n$ such that $15120 n$ is a perfect square. (\textbf{Hint:} How could you identify a perfect square if you were able to see its PPF?)
\end{problem}

\begin{solution}
    First we want to find the PPF of $15120$, we can find that as follows,
    \[
        \begin{matrix}
            15120   &                                           &          \\
                    & 15120                      \equiv 0       & \pmod{5} \\
            \\
            15120 = & \multicolumn{2}{l}{5 \cdot 3024}                     \\
            \\
                    & 3024                             \equiv 0 & \pmod{4} \\
            \\
            15120 = & 5 \cdot 2^2 \cdot 756                     &          \\
            \\
                    & 756                             \equiv 0  & \pmod{4} \\
            \\

            15120 = & 5 \cdot 2^4 \cdot 189                                \\
            \\
                    & 189                             \equiv 0  & \pmod{9} \\
            \\
            15120 = & 5 \cdot 2^4 \cdot 3^2 \cdot 21                       \\
        \end{matrix}
    \]
    So the PPF is $2^4 \cdot 3^3 \cdot 5\cdot 7$, the smallest $n$ that would make $2^4 \cdot 3^3 \cdot 5\cdot 7\cdot n$ a perfect square would be $n=3\cdot 5 \cdot 7$ or $n=105$.

    This would make $15120n = 1589600$, which is $1260^2$.
\end{solution}


\bline
\section{(Page 26)}

\begin{problem}{4}
Find all the solutions in positive integers of $2x+y=2$, $3x-4y=0$, and $7x+15y=51$.
\end{problem}

\begin{solution}
    \begin{itemize}
        \item [(a)] $2x+y=2$ \\
              One solution we can see by inspection is $x=2$ and $y=-2$. Then, all solutions will be
              \begin{align*}
                  x = 2 + \frac{1}{(1,2)}t & ~ & y = -2 + \frac{2}{(1,2)}t \\
                  x = 2 + t                & ~ & y = -2+ 2t                \\
              \end{align*}
        \item [(b)] $3x-4y=0$ \\
              One solution we can see by inspection is $x=4$ and $y=-3$. Then, all solutions will be
              \begin{align*}
                  x = 4 + \frac{4}{(3,4)}t & ~ & y = -3 + \frac{3}{(3,4)}t \\
                  x = 4 + 4t               & ~ & y = -3 + 3t               \\
              \end{align*}
        \item [(c)] $7x+15y=51$ \\
              One solution we can see by inspection of $7x+15y=1$ would be $x=2$ and $y=-1$, so one solution to $7x+15y=51$ would be $x=102$ and $y=-51$. Then, all solutions will be
              \begin{align*}
                  x = 102 + \frac{15}{(7,15)}t & ~ & y = -51 + \frac{7}{(7,15)}t \\
                  x = 102 + 15t                & ~ & y = -51 + 7t                \\
              \end{align*}

    \end{itemize}
\end{solution}


\bline
\section{(Pages 32 - 33)}

\begin{problem}{2}
Find the least residue of $1789 \pmod{4},\pmod{10},\text{and} \pmod{101}$.
\end{problem}

\begin{solution}
    \begin{itemize}
        \item $1789 \pmod{4}$
              \begin{align*}
                  1789 & = 1600 + 189                     \\
                  1789 & \equiv 0 + 160 + 29   & \pmod{4} \\
                  1789 & \equiv 0 + 0 + 28 + 1 & \pmod{4} \\
                  1789 & \equiv  1             & \pmod{4} \\
              \end{align*}
        \item $1789 \pmod{10}$
              \begin{align*}
                  1789 & = 1700 + 89                   \\
                  1789 & \equiv 0 + 80 + 9 & \pmod{10} \\
                  1789 & \equiv  9         & \pmod{10} \\
              \end{align*}
        \item $1789 \pmod{101}$
              \begin{align*}
                  1789 & = 1717 + 72                \\
                  1789 & \equiv 0 + 72 & \pmod{101} \\
                  1789 & \equiv 72     & \pmod{101} \\
              \end{align*}
    \end{itemize}
\end{solution}



\begin{problem}{6}
Find all $m$ such that $1848 \equiv 1914 \pmod{m}.$
\end{problem}

\begin{solution}
    $1848 \equiv 1914 \pmod{m}$ iff $1914=1848 + km$ for some $k\in\Z$. This means that we need $66=km$. The PPD of 66 is $11\cdot 3 \cdot 2$, so $m$ can be in $\{2,3,6,11,22,33,66\}$.
\end{solution}


\begin{problem}{8}
Show that every prime (except 2) is congruent to 1 or 3$\pmod{4}$.
\end{problem}

\begin{myproof}
    Let $p$ be any prime (other than 2). By definition we know that $2 \ndiv p$. We also can see that for $a\equiv 2\pmod{4}$ or $a\equiv 0\pmod{4}$ that either $a=2 +4k$ or $a=4k$ for some $k$. No matter what $k$ we choose, $2|4k$, so $2|a$ must also be true. Therefore we know that $p\neq a$, so $p$ must be congruent to either 1 or 3$\pmod{4}$.
\end{myproof}


\begin{problem}{9}
Show that every prime (except 2 or 3) is congruent to 1 or 5$\pmod{6}$.
\end{problem}

\begin{myproof}
    Let $p$ be any prime (other than 2 or 3). By definition $2\ndiv a$ and $3\ndiv a$. For $a$ to be congruent to 2, 3, 4, or 0, then one of the following must be true: $a=6k$, $a=2 +6k$, $a=3 +6k$, or $a=4 +6k$.
    Then we can see that either $2|a$ ($a=6k$, $a=2+6k$, or $a=4+6k$), or $3|a$ ($a=6k$ or $a=3+6k$). Therefore $a\neq p$. So $p$ must be congruent to 1 or 5$\pmod{6}$.
\end{myproof}



\begin{problem}{10}
What can primes (except 2,3, or 5) be congruent to$\pmod{30}$?
\end{problem}

\begin{solution}
    Let $p$ be any prime (other than 2,3, or 5). Then, $p$ must be congruent to $a$ where $a\neq k$ for $k\bot 30$. So $p$ must be congruent to something in the set $\{1, 7, 11, 13, 17, 19, 23, 29\}$.
\end{solution}


\begin{problem}{11}
In the multiplication $31415\cdot92653=2910\underline{\phantom{3}}93995$, one digit in the product is missing and all the others are correct. Find the missing digit without doing the multiplication.
\end{problem}

\begin{solution}
    We can see that $11|92653$ by the 11 division prop, because $9-2+6-5+3=11$. This means that we know that our answer must also be divisible by 11.
    Subbing in $x$ for our missing digit we can get that
    \[
        2+9-1+0-x+9-3+9-9+5 = 21-x
    \]
    For $21-x\equiv 0 \pmod{11}$ and $0\leq x\leq 9$, we see that $x=9$.
    So our missing digit is 9.
\end{solution}


\begin{problem}{14}
Show that the difference of two consecutive cubes is never divisible by 3.
\end{problem}

\begin{myproof}
    Let $x=a^3$ and $y=(a+1)^3$. Then
    \begin{align*}
        y - x & = (a+1)^3 - a^3           \\
              & = a^3 - 3a^2 + 3a -1 -a^3 \\
              & = - 3a^2 + 3a -1          \\
              & \equiv - 1 \pmod{3}       \\
    \end{align*}
    This means that the difference two consecutive cubes will always be equivalent to -1 mod 3.
\end{myproof}


\begin{problem}{15}
Show that the difference of two consecutive cubes is never divisible by 5.
\end{problem}

\begin{myproof}
    Let $x=a^3$ and $y=(a+1)^3$. Then
    \begin{align*}
        y - x & = (a+1)^3 - a^3                \\
              & = a^3 - 3a^2 + 3a -1 -a^3      \\
              & = - 3a^2 + 3a -1               \\
              & \equiv - 3a^2 + 3a -1 \pmod{5} \\
    \end{align*}
    Assume to the contrary that it is divisible by 5, then
    \begin{align*}
        -3a^2 + 3a -1 & \equiv 0   \pmod{5} \\
        -3a^2 + 3a    & \equiv 1   \pmod{5} \\
        -3(a^2 -a)    & \equiv 1   \pmod{5} \\
        2(a^2 -a)     & \equiv 1   \pmod{5} \\
    \end{align*}
    Then we can make the chart:

    \setlength{\tabcolsep}{3pt}
    \begin{center}
        \begin{tabular}{r||c|c|c|c|c}
            $a     \pmod{5}$    & $0\pmod{5}$ & $1\pmod{5}$ & $2\pmod{5}$ & $3\pmod{5}$  & $4\pmod{5}$    \\
            $a^2  \pmod{5}$     & 0           & 1           & 4           & $9 \equiv 4$ & $16 \equiv 1 $ \\
            $a^2-a \pmod{5}$    & 0           & 0           & 2           & 1            & $-3 \equiv 2 $ \\
            $2(a^2-a) \pmod{5}$ & 0           & 0           & 4           & 2            & 4              \\
        \end{tabular}
    \end{center}
    We can see that none of these are able to be congruent to 1, so we can see that this is a contradiction so the difference of two consecutive cubes is never divisible by 5.

\end{myproof}


\begin{problem}{19}
Show that if $n\equiv 4\pmod{9}$, then $n$ cannot be written as the sum of three cubes.
\end{problem}

\begin{solution}
    We can make the chart:

    \setlength{\tabcolsep}{3pt}
    \begin{center}
        \begin{tabular}{r||c|c|c|c|c|c|c|c|c}
            $r    \pmod{9}$    & 0 & 1 & 2 & 3 & 4 & 5 & 6 & 7 & 8 \\
            $r^2     \pmod{9}$ & 0 & 1 & 4 & 0 & 7 & 7 & 0 & 4 & 1 \\
            $r^3     \pmod{9}$ & 0 & 1 & 8 & 0 & 1 & 8 & 0 & 1 & 8 \\
        \end{tabular}
    \end{center}

    We can see that cubes can only be congruent to 0, 1 or 8 in mod 9. So the only sums that three cubes can get to are 0, 1, 2, 3, 8, $9\equiv 0$, $10\equiv 1$, $16\equiv 5$, $17 \equiv 6$, or $24 \equiv 6$. So there is no way for the sum of three cubes to be congruent to 4 mod 9.
\end{solution}

\bline
\section{(Pages 40 - 41)}

\begin{problem}{1}
Solve each of the following:
\begin{itemize}
    \item [(a)] $2x\equiv 1 \pmod{17}$
    \item [(b)] $3x\equiv 1 \pmod{17}$
    \item [(c)] $3x\equiv 6 \pmod{18}$
    \item [(d)] $40x\equiv 777 \pmod{1777}$
\end{itemize}

\end{problem}

\begin{solution}
    \begin{itemize}
        \item [(a)] $(2,17)=1$, so there is only 1 solution and it is $x\equiv9\pmod{17}$.
        \item [(b)] Since $(3,17)=1$, there is only 1 solution and it is $x\equiv6\pmod{17}$.
        \item [(c)] Since $(3,18)=3$, and $3|6$, we can rewrite it as $x\equiv 2 \pmod{6}$. Then, there are 3 solutions, and they are $x\equiv 2, 8, 14 \pmod{18}$.
        \item [(d)] $(40,1777)=1$, so there is only 1 solution, and we can use EA to solve as follows,
              \begin{align*}
                  1777 & = 40(44)+17 \\
                  40   & = 17(2) + 6 \\
                  17   & = 6(2) + 5  \\
                  6    & = 5(1) + 1  \\
                  5    & = 1(5)+0    \\
              \end{align*}
              Then using back substitution,
              \begin{align*}
                  1   & = 6 - 5                      \\
                      & = 6 - 17+6(2)                \\
                      & = 17(-1) + 6(3)              \\
                      & = 17(-1) + 3(40-17(2))       \\
                      & = 17(-1) + 40(3) + 17(-6)    \\
                      & = 17(-7) + 40(3)             \\
                      & = (-7)(1777-40(44)) + 40(3)  \\
                      & = 1777(-7) + 40(308) + 40(3) \\
                  1   & = 1777(-7) + 40(311)         \\
                  777 & = 1777(-5439) +40(241647)    \\
                  777 & = 40(241647) \pmod{1777}     \\
                  777 & = 40(1752) \pmod{1777}       \\
              \end{align*}
              So we can see that $x\equiv 1752\pmod{1777}$
    \end{itemize}
\end{solution}


\begin{problem}{3}
Solve the systems
\begin{itemize}
    \item [(a)] $x\equiv 1 \pmod{2},\; x\equiv 1 \pmod{3}$.
    \item [(b)] $x\equiv 3 \pmod{5}, \; x\equiv 5\pmod{7}, \; x \equiv 7 \pmod{11}$.
    \item [(c)] $2x\equiv 1 \pmod{5}, \; 3x\equiv 2 \pmod{7}, \; 4x\equiv 3 \pmod{11}$.
\end{itemize}
\end{problem}

\begin{solution}
    \begin{itemize}
        \item [(a)] Let $k_1, k_2\in \Z$. Then we can do the following,
              \[
                  \begin{matrix*}[l]
                      x \equiv 1 \pmod{2}        & \rightarrow & x= 2k_1 + 1      \\
                      2k_1 + 1 \equiv 1 \pmod{3} & \leftarrow  &                  \\
                      2k_1 \equiv 0 \pmod{3}     &                                \\
                      k_1 \equiv 0 \pmod{3}      & \rightarrow & k_1 = 3k_2       \\
                      &             & x= 2(3(k_2)) + 1 \\
                      &             & x= 6k_2 + 1      \\
                      x \equiv 1 \pmod{6}        & \leftarrow  &                  \\
                  \end{matrix*}
              \]
        \item [(b)] Let $k_1, k_2, k_3 \in \Z$. Then we can do the following,
              \[
                  \begin{matrix*}[l]
                      x \equiv 3 \pmod{5}           & \rightarrow & x= 5k_1 + 3           \\
                      5k_1 + 3 \equiv 5 \pmod{7}    & \leftarrow  &                       \\
                      5k_1 \equiv 2 \pmod{7}        &             &                       \\
                      k_1 \equiv 6 \pmod{7}         & \rightarrow & k_1= 7k_2 + 6         \\
                      &             & x= 5(7k_2 + 3) + 6    \\
                      &             & x= 35k_2 + 15 + 6     \\
                      &             & x= 35k_2 + 21         \\
                      35k_2 + 21 \equiv 7 \pmod{11} & \leftarrow  &                       \\
                      2k_2 - 1 \equiv 7 \pmod{11}   &             &                       \\
                      2k_2 \equiv 8 \pmod{11}       &             &                       \\
                      k_2 \equiv 4 \pmod{11}        & \rightarrow & k_2 = 11k_3 + 4       \\
                      &             & x= 35(11k_3 + 4) + 21 \\
                      &             & x= 385k_3 + 140 + 21  \\
                      &             & x= 385k_3 + 161       \\
                      x \equiv 161 \pmod{385}       & \leftarrow  &
                  \end{matrix*}
              \]
        \item [(c)] [(b)] Let $k_1, k_2, k_3 \in \Z$. Then we can do the following,
              \[
                  \begin{matrix*}[l]
                      2x \equiv 1 \pmod{5}          &             &                 \\
                      x \equiv 3 \pmod{5}           & \rightarrow & x= 5k_1 + 3     \\
                      3(5k_1 + 3) \equiv 2 \pmod{7} & \leftarrow  &                 \\
                      15k_1 + 9 \equiv 2 \pmod{7}   &             &                 \\
                      k_1 \equiv 0 \pmod{7}         & \rightarrow & k_1 = 7k_2      \\
                      &             & x= 5(7k_2) +3   \\
                      &             & x= 35k_2 + 3    \\
                      4(35k_2+3) \equiv 3 \pmod{11} & \leftarrow  &                 \\
                      k_2 + 1 \equiv 3 \pmod{11}    &             &                 \\
                      k_2 \equiv 2 \pmod{11}        & \rightarrow & k_2 = 11k_3 + 2 \\
                      &             & x= 35(11k_3+2)  \\
                      &             & x= 385k_3 + 70  \\
                      x\equiv 70 \pmod{385}         & \leftarrow  &                 \\
                  \end{matrix*}
              \]
    \end{itemize}
\end{solution}


\begin{problem}{5}
What possibilities are there for number of solutions of a linear congruence (mod 20)
\end{problem}

\begin{solution}
    The possibilities are any possibilities of $(a,20)$, which can be anything in the set $\{0,1,2,4,5,10,20\}$
\end{solution}


\begin{problem}{6}
Construct linear congruences modulo 20 with no solutions, just one solution, and more than one solution. Can you find one with 20 solutions?
\end{problem}

\begin{solution}
    \[
        \begin{matrix*}[l]
            \text{\textbf{No Solutions:}} &                  \\
            & ax \equiv b \pmod{20} &\text{ where } (a,m)\ndiv b \\
            \text{\textbf{\phantom{No Solutii}Ex:}} &                  \\
            & 5x \equiv 7 \pmod{20} \\
            \text{\textbf{  1 Solutions:}} &                  \\
            & ax \equiv b \pmod{20} &\text{ where } (a,m)=1 \\
            \text{\textbf{\phantom{  1 Solutii}Ex:}} &                  \\
            & 7x \equiv 13 \pmod{20} \rightarrow & x\equiv 19 \pmod{20} \\
            \text{\textbf{  k Solutions:}} &                  \\
            & ax \equiv b \pmod{20} &\text{ where } (a,m)=k \text{ and } k|b \\
            \text{\textbf{\phantom{  k Sii}Ex (2):}} &                  \\
            & 6x \equiv 14 \pmod{20} \rightarrow & x\equiv 9, 19 \pmod{20} \\
            \text{\textbf{\phantom{  k Sii}Ex (4):}} &                  \\
            & 8x \equiv 16 \pmod{20} \rightarrow&  x\equiv 2, 6, 10, 16 \pmod{20} \\
            \text{\textbf{\phantom{  k Sii}Ex (5):}} &                  \\
            & 15x \equiv 5 \pmod{20} \rightarrow & x\equiv 3, 7, 11, 15, 19 \pmod{20} \\
            \text{\textbf{\phantom{  k S}Ex (10):}} &                  \\
            & 10x \equiv 10 \pmod{20} \rightarrow & x\equiv 1, 3, 5, 7,9,11,13,15,17,19\pmod{20} \\
            \text{\textbf{\phantom{  k S}Ex (20):}} &                  \\
            & 20x \equiv 20 \pmod{20} \rightarrow & x\equiv 0,1,2,3,4,5,6,7,8,9,10,11\\
            && \phantom{x\equiv' } 12,13,14,15,16,17,18,19 \pmod{20} \\

        \end{matrix*}
    \]
\end{solution}


\bline

\section{}
\begin{problem}{}
Let $f(x) = x^2+x+41$.
\begin{itemize}
    \item [(a)] Have Sage compute $f(n)$ for $n=1,2,\cdots, 10$ and make a conjecture about the possible values of $f(n)$ when $n$ is any positive integer
    \item [(b)] Prove or disprove your conjecture from part (a).
    \item [(c)] \textbf{Extra Credit:} Prove that for any polynomial of the form $f(x)=ax^2+bx+c$ with $a,b,c\in\Z$ and $a\neq 0$, $f(n)$ will be \textit{composite} for infinitely many positive integers $n$.
    \item [(d)] \textbf{Extra Credit:} Prove that you can find a non-constant quadratic polynomial $f(x)$ such that $f(n)$ is prime for infinitely many values of $n$. (\textbf{Hint:} Do the rest of your homework first)
\end{itemize}
\end{problem}

\begin{solution}
    \begin{itemize}
        \item [(a)]
              \lstinputlisting[language=Python, firstline=2, lastline =26]{Homework2frozen.md}

              \lstinputlisting[firstline=29, lastline =40]{Homework2frozen.md}
              My conjecture is that $f(n)$ will be prime for all $n$.
        \item [(b)]
              \lstinputlisting[language=Python, firstline=44, lastline =61]{Homework2frozen.md}

              \lstinputlisting[firstline=64, lastline =64]{Homework2frozen.md}
        \item [(c)] Let $f(x)=ax^2+bx+c$ with $a,b,c\in\Z$ and $a\neq 0$. Assume for contradiction that $f(n)$ for all $n$ be prime. Then let $n=c$. Then we can see
              \begin{align*}
                  f(c) & = a(c)^2 + b(c) + c             \\
                       & \equiv ac^2 + b(c) + c \pmod{c} \\
                       & \equiv 0 \pmod{c}               \\
              \end{align*}
              By definition this means that $f(c)= c\cdot k$ for some $k$ but this means that $c|f(c)$, which means that $f(c)$ is not prime, so this is a contradiction. We can also see that this will be true for any $f(n)$ where $c|n$.

              So we can see that $f(n)$ will be composite for infinitely many positive integers $n$.
        \item [(d)]
    \end{itemize}
\end{solution}








\end{document}
